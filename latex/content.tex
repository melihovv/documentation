\section{Краткое описание темы индивидуальной работы}
Программа предназначена для классификации набора записей по набору классов, в 
которых принадлежность записи определяется путем проверки набора правил.

\section{Диаграмма классов}
\melimg{class_diagram}{Диаграмма классов.}

\section{Результаты профилирования}
Как видно из рисунка \ref{img:profile_result}, наибольшее количество времени программа проводит в функциях из библиотеки Qt5. Первая из них — ucstrncmp используется для сравнения названий правил и записей, вторая — для создания строк.
\melimg{profile_result}{Результаты профилирования программы.}

\section{Выводы по результатам профилирования}
Так как вышеуказанные функции занимают малую долю работы программы, здесь нет необходимости в оптимизировании кода.
